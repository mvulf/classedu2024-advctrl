\documentclass[12pt]{article}
\usepackage{amssymb,amsfonts}
\usepackage{mathtools}
\usepackage{amsmath}
\usepackage{algorithmic}
\usepackage{graphicx}
\usepackage{textcomp}
\usepackage{float}
\usepackage{float}
\usepackage{wrapfig}
\usepackage{xcolor}
\usepackage{microtype}
\usepackage{booktabs}
\usepackage{cleveref}
\input{preamble.tex}

\title{Advanced Control Methods 2024. Homework 1. Energy-Based Control for Cartpole System}


\begin{document}

\section*{Cartpole system}

The dynamics of the cartpole system are governed by the following set of differential equations:

\begin{equation}
    \label{eqn_state_dynamics}
    \begin{aligned}
        &\dot{\vartheta} =  \omega \\
        &\dot{x} = v_x \\
        &\dot{\omega} =  \frac{g \sin{\vartheta}(m_c + m_p) - \cos{\vartheta}(F + m_p l \omega^2 \sin{\vartheta})}{\frac{4l}{3}(m_c + m_p) - lm_p \cos^2{\vartheta}}\\
        &\dot{v}_x = \frac{F + m_p l (\omega ^2 \sin{\vartheta} - \dot{\omega}   \cos{\vartheta})}{m_c + m_p}
    \end{aligned}
\end{equation}
where the variables are defined as follows:
\begin{itemize}
\item $\vartheta$: pole turning angle (\textbf{state variable}) [rad]
\item $x$: x-coordinate of the cart (\textbf{state variable}) [m]
\item $\omega$: pole angular speed with respect to relative coordinate axes with cart in the origin (\textbf{state variable}) [rad/s]
\item $v_x$: absolute speed of the cart (\textbf{state variable}) [m/s]
\item $F$: pushing force (\textbf{control variable}) [N]
\item $m_c$: mass of the cart [kg]
\item $m_p$: mass of the pole [kg]
\item $l$: pole length [m]
  
\end{itemize}
Lagrange's equations are employed to derive the above expressions \eqref{eqn_state_dynamics}:

\begin{eqnarray}
\label{eqn_sum_moments}
& \overline{I}_p \dot{\omega} + \dot{v}_x m_p l\cos \vartheta  - m_p g l \sin \vartheta = 0 \\ 
\label{eqn_2nd_newton_law}
& (m_c + m_p) \dot{v}_x - m_p l v_x^2 \sin \vartheta + \dot{\omega} m_p l \cos \vartheta  = F
\end{eqnarray}
with the moment of inertia $\overline{I}_p$ given by $\overline{I}_p  = \frac{4}{3}m_p l ^ 2 $.


\section*{Exercise 1}

Define the pendulum's energy as:

$$
E_p =  \overline{I}_p  \omega ^ 2 + m g l (\cos{\vartheta} - 1)
$$
Consider the function:

\begin{equation}
    L = \frac{1}{2}(E_p^2 + ml\lambda v_x ^ 2)
\end{equation}
Prove that the $L$ time derivative is: 
\begin{equation}
\label{eqn_derivative_lyapunov}
\frac{\diff L}{\diff t} = - \dot{v}_x m l (E_p \omega \cos \vartheta - \lambda v_x)
\end{equation}
\textbf{Hint.} \textit{You will need equation \eqref{eqn_sum_moments} to derive \eqref{eqn_derivative_lyapunov}.}

\section*{Exercise 2}

By substituting:
\begin{equation}
\label{eqn_energy_based_v_x}
\dot{v}_x = k (E_p \omega \cos \theta - \lambda v_x)
\end{equation}
into \eqref{eqn_derivative_lyapunov}, we obtain:
$$
\frac{\diff L}{\diff t} = - mlk (E_p\omega \cos \theta  - \lambda v_x)^2 \leq 0,
$$
Find such a value of control variable $F$ that enforces the above condition \eqref{eqn_energy_based_v_x} for $\dot{v}_x$. 
This will constitute the energy-based control law. Note that $k\in \R$ is a positive constant hyperparameter.
\\
\\
\textbf{Hint.} \textit{You will need equations \eqref{eqn_sum_moments} and \eqref{eqn_2nd_newton_law} to enforce \eqref{eqn_energy_based_v_x}.}

\end{document}