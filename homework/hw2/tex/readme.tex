\documentclass[12pt]{article}
\usepackage{amssymb,amsfonts}
\usepackage{mathtools}
\usepackage{amsmath}
\usepackage{algorithmic}
\usepackage{graphicx}
\usepackage{textcomp}
\usepackage{float}
\usepackage{float}
\usepackage{wrapfig}
\usepackage{xcolor}
\usepackage{microtype}
\usepackage{booktabs}
\usepackage{cleveref}
%%%%%%%%%%%%%%%%%%%%%%%%%%%%%%%%%%%%%%%%%%%%%%%%%%%%%%%%%%%%%%%%%%%%%%%%%%%%%%%%%%%%%%%%%%%%%
%%%%%%%%%%%%%%%%%%%%%%%%%%%%%%%%%%%%%%%%%%%%%%%%%%%%%%%% REQUIREMENTS
% color
% xspace (!)
% ams packages
% mathspec
% mathtools
%%%%%%%%%%%%%%%%%%%%%%%%%%%%%%%%%%%%%%%%%%%%%%%%%%%%%%%%%%%%%%%%%%%%%%%%%%%%%%%%%%%%%%%%%%%%
%%%%%%%%%%%%%%%%%%%%%%%%%%%%%%%%%%%%%%%%%%%%%%%%%%%%%%% GENERAL SYMBOLS
\newcommand{\diff}{\mathop{}\!\mathrm{d}}								% Differential
\newcommand{\pdiff}[2]{ { \frac{\partial {#1}}{\partial {#2}} } }		% Partial differentiation
\newcommand{\D}{\ensuremath{\mathcal{D}}}								% Generalized derivative
\newcommand{\eps}{{\varepsilon}}										% Epsilon
\newcommand{\ball}{{\mathcal B}}										% Ball
\newcommand{\clip}{{\text{clip}}}										% Clip function
\newcommand{\Lip}[1]{\text{Lip}_{#1}}									% Lipschitz constant of #1
\newcommand{\sgn}{{\text{sgn}}}											% Signum function
\newcommand{\diam}{{\text{diam}}}										% Diameter
\newcommand{\dom}{{\text{dom}}}											% Domain
\newcommand{\ramp}{{\text{ramp}}}										% Ramp	
\newcommand{\co}{{\overline{\text{co}}}}								% Convex closure
\DeclareMathOperator*{\argmin}{\text{arg\,min}}							% Argmin
\DeclareMathOperator*{\argmax}{\text{arg\,max}}							% Argmax
%\newcommand{\ln}{\text{ln}}												% Natural logarithm
\newcommand{\transp}{\ensuremath{^{\top}}}								% Matrix transpose
\newcommand{\inv}{\ensuremath{^{-1}}}									% Inverse
\newcommand{\tovec}[1]{\ensuremath{\text{vec}}\left(#1\right)}			% To-vector transformation
\newcommand{\nrm}[1]{\left\lVert#1\right\rVert}							% Norm
\newcommand{\diag}[1]{{\text{diag}}\left(#1\right)}						% Diagonal
\newcommand{\abs}[1]{\left\lvert#1\right\rvert}							% Absolute value
\newcommand{\scal}[1]{\left\langle#1\right\rangle}						% Scalar product
\newcommand{\tr}[1]{{\text{tr}}\left(#1\right)}							% Trace
\newcommand{\E}[2][{}]{\mathbb E_{#1}\left[#2\right]}					% Mean
\newcommand{\Es}[2][{}]{\hat {\mathbb E}_{#1}\left[#2\right]}			% Sample mean
\newcommand{\PP}[1]{\mathbb P\left[#1\right]}							% Probability
\newcommand{\bigo}[1]{\mathcal O\left(#1\right)}						% Big-o
\newcommand{\low}{{\text{low}}}											% Lower bound
\newcommand{\up}{{\text{up}}}											% Upper bound
%\newcommand\circled[1]{\tikz[baseline=(char.base)]{\node[shape=circle,draw,inner sep=1pt](char){#1};}}
%%%%%%%%%%%%%%%%%%%%%%%%%%%%%%%%%%%%%%%%%%%%%%%%%%%%%%%%%%%%%%%%%%%%%%%%%%%%%%%%%%%%%%%%%%%%%
%%%%%%%%%%%%%%%%%%%%%%%%%%%%%%%%%%%%%%%%%%%%%%%%%%%%%%%% ARROWS
\newcommand{\ra}{\rightarrow}											% Right arrow
\newcommand{\la}{\leftarrow}											% Left arrow
\newcommand{\rra}{\rightrightarrows}									% Double right arrow
%%%%%%%%%%%%%%%%%%%%%%%%%%%%%%%%%%%%%%%%%%%%%%%%%%%%%%%%%%%%%%%%%%%%%%%%%%%%%%%%%%%%%%%%%%%%
%%%%%%%%%%%%%%%%%%%%%%%%%%%%%%%%%%%%%%%%%%%%%%%%%%%%%%% ABBREVIATIONS
\newcommand{\ie}{\unskip, i.\,e.,\xspace}								% That is
\newcommand{\eg}{\unskip, e.\,g.,\xspace}								% For example
\newcommand{\sut}{\text{s.\,t.\,}}										% Such that or subject to
\newcommand{\wrt}{w.\,r.\,t. \xspace}									% With respect to
%%%%%%%%%%%%%%%%%%%%%%%%%%%%%%%%%%%%%%%%%%%%%%%%%%%%%%%%%%%%%%%%%%%%%%%%%%%%%%%%%%%%%%%%%%%%%
%%%%%%%%%%%%%%%%%%%%%%%%%%%%%%%%%%%%%%%%%%%%%%%%%%%%%%% SETS
\let\oldemptyset\emptyset
\let\emptyset\varnothing
\newcommand{\N}{{\mathbb{N}}}											% Set of natural numbers
\newcommand{\Z}{{\mathbb{Z}}}											% Set of integer numbers
\newcommand{\Q}{{\mathbb{Q}}}											% Set of rational numbers
\newcommand{\R}{{\mathbb{R}}}											% Set of real numbers
%%%%%%%%%%%%%%%%%%%%%%%%%%%%%%%%%%%%%%%%%%%%%%%%%%%%%%%%%%%%%%%%%%%%%%%%%%%%%%%%%%%%%%%%%%%%
%%%%%%%%%%%%%%%%%%%%%%%%%%%%%%%%%%%%%%%%%%%%%%%%%%%%%%%% COLORED
%\newcommand{\red}[1]{\textcolor{red}{#1}}
%\newcommand{\blue}[1]{\textcolor{blue}{#1}}
%\definecolor{dgreen}{rgb}{0.0, 0.5, 0.0}
%\newcommand{\green}[1]{\textcolor{dgreen}{#1}}
%%%%%%%%%%%%%%%%%%%%%%%%%%%%%%%%%%%%%%%%%%%%%%%%%%%%%%%%%%%%%%%%%%%%%%%%%%%%%%%%%%%%%%%%%%%%
%%%%%%%%%%%%%%%%%%%%%%%%%%%%%%%%%%%%%%%%%%%%%%%%%%%%%%% SYSTEMS AND CONTROL
\newcommand{\state}{x}													% State (as vector)
\newcommand{\State}{X}													% State (as random variable)
\newcommand{\states}{\mathbb X}											% State space
\newcommand{\action}{u}													% Action (as vector)	
\newcommand{\Action}{U}													% Action (as random variable)
\newcommand{\actions}{\mathbb U}										% Action space
\newcommand{\traj}{z}													% State-action tuple (as vector tuple)
\newcommand{\Traj}{Z}													% State-action tuple (as random variable tuple)
\newcommand{\obs}{y}													% Observation (as vector)
\newcommand{\Obs}{Y}													% Observation (as random variable)
\newcommand{\obses}{\mathbb Y}											% Observation space
\newcommand{\policy}{\rho}												% Policy (as function or distribution)
\newcommand{\policies}{\mathcal U}										% Policy space
\newcommand{\transit}{f}												% State transition map
\newcommand{\reward}{r}													% Reward (as vector)
\newcommand{\Reward}{R}													% Reward (as random varaible)
\newcommand{\cost}{c}													% Cost (as vector)
\newcommand{\Cost}{C}													% Cost (as random varaible)
\newcommand{\Value}{V}													% Value
\newcommand{\Advan}{\mathcal A}											% Advantage
\newcommand{\W}{\ensuremath{\mathbb{W}}}								% Weight space
\newcommand{\B}{\ensuremath{\mathbb{B}}}								% Basin
\newcommand{\G}{\ensuremath{\mathbb{G}}}								% Attractor (goal set)
\newcommand{\Hamilt}{\ensuremath{\mathcal{H}}}							% Hamiltonian
\newcommand{\K}{\ensuremath{\mathcal{K}}\xspace}						% Class kappa
\newcommand{\KL}{\ensuremath{\mathcal{KL}}\xspace}						% Class kappa-ell
\newcommand{\Kinf}{\ensuremath{\mathcal{K}_{\infty}}\xspace}			% Class kappa-infinity
\newcommand{\KLinf}{\ensuremath{\mathcal{KL}_{\infty}}\xspace}			% Class kappa-ell-infinity
\newcommand{\T}{\mathcal T}												% Total time
\newcommand{\deltau}{\Delta \tau}										% Time step size
\newcommand{\dt}{\ensuremath{\mathrm{d}t}}								% Time differential
\newcommand{\normpdf}{\ensuremath{\mathcal N}}							% Normal PDF
\newcommand{\trajpdf}{p}												% State-action PDF
\newcommand{\TD}{\delta}												% Temporal difference
\newcommand{\old}{{\text{old}}}											% Old (previous) index
\newcommand{\loss}{\mathcal L}											% Loss
\newcommand{\replay}{\mathcal R}										% Experience replay
\newcommand{\safeset}{\ensuremath{\mathcal{S}}}							% Safe set
\newcommand{\dkappa}{\kappa_{\text{dec}}}								% Decay kappa function



\title{Advanced Control Methods 2024. Homework 1. Energy-Based Control for Cartpole System}


\begin{document}

\section*{Cartpole system with friction}

The dynamics of the cartpole system with friction are governed by the following set of differential equations:

\begin{equation}
    \label{eqn_state_dynamics}
    \begin{aligned}
        &\dot{\vartheta} =  \omega \\
        &\dot{x} = v_x \\
        &\dot{\omega} =  \frac{g \sin{\vartheta}(m_c + m_p) - \cos{\vartheta}(F - b v_x + m_p l \omega^2 \sin{\vartheta})}{\frac{4l}{3}(m_c + m_p) - lm_p \cos^2{\vartheta}}\\
        &\dot{v}_x = \frac{F - b v_x + m_p l \omega ^2 \sin{\vartheta} - \frac{3}{8}m_p g\sin(2\vartheta)}{m_c + m_p - \frac{3}{4} m_p \cos ^ 2 \vartheta}
    \end{aligned}
\end{equation}
where the variables are defined as follows:
\begin{itemize}
\item $\vartheta$: pole turning angle (\textbf{state variable}) [rad]
\item $x$: x-coordinate of the cart (\textbf{state variable}) [m]
\item $\omega$: pole angular speed with respect to relative coordinate axes with cart in the origin (\textbf{state variable}) [rad/s]
\item $v_x$: absolute speed of the cart (\textbf{state variable}) [m/s]
\item $F$: pushing force (\textbf{control variable}) [N]
\item $m_c$: mass of the cart [kg]
\item $m_p$: mass of the pole [kg]
\item $l$: pole length [m]
\item $b$: friction coefficient [kg/s]  
\end{itemize}
Lagrange's equations are employed to derive the above expressions \eqref{eqn_state_dynamics}:

\begin{eqnarray}
\label{eqn_sum_moments}
& \overline{I}_p \dot{\omega} + \dot{v}_x m_p l\cos \vartheta  - m_p g l \sin \vartheta = 0 \\ 
\label{eqn_2nd_newton_law}
& (m_c + m_p) \dot{v}_x - m_p l \omega^2 \sin \vartheta + \dot{\omega} m_p l \cos \vartheta  + b v_x = F
\end{eqnarray}
with the moment of inertia $\overline{I}_p$ given by $\overline{I}_p  = \frac{4}{3}m_p l ^ 2 $.
\\
\\
The variable correspondences in the code are as follows:
\begin{itemize}
    \item $\vartheta=$\texttt{angle}
    \item $\omega=$\texttt{angle\_vel}
    \item $v_x=$\texttt{vel}
    \item $b=$\texttt{friction\_coeff} 
    \item $m_c=$\texttt{mass\_cart}
    \item $m_p=$\texttt{mass\_pole}
    \item $l=$\texttt{length\_pole}
    \item $g=$\texttt{grav\_const}
\end{itemize}
\section*{Exercise 1}
\subsection*{Theory}
Define the pendulum's energy as:

$$
E_p =  \frac{\overline{I}_p \omega ^ 2}{2}   + m_p g l (\cos{\vartheta} - 1)
$$
Consider the function:

\begin{equation}
    L_1 = \frac{1}{2}(E_p^2 + m_p l\lambda v_x ^ 2),
\end{equation}
where $\lambda \in \R_{> 0}$ is a positive constant hyperparameter. 
Prove that the $L_1$ time derivative is: 
\begin{multline}
\label{eqn_derivative_lyapunov}
\frac{\diff L_1}{\diff t} = - \dot{v}_x m_p l (E_p \omega \cos \vartheta - \lambda v_x) = \\ 
= -\frac{F - b v_x + m_p l \omega ^2 \sin{\vartheta} - \frac{3}{8}m_p g\sin(2\vartheta)}{m_c + m_p - \frac{3}{4} m_p \cos ^ 2 \vartheta} m_p l (E_p \omega \cos \vartheta - \lambda v_x)
\end{multline}
and identify such a control function $F = F_{\text{fr.comp.}}(\vartheta, \omega, v_x, b)$ that ensures 
$$
\frac{\diff L_1}{\diff t} = - m_p lk (E_p\omega \cos \vartheta  - \lambda v_x)^2,
$$
where $k\in \R_{> 0}$ is a positive constant hyperparameter.
\\
\\
\textbf{Hint.} \textit{You will need equation \eqref{eqn_sum_moments} to derive \eqref{eqn_derivative_lyapunov}.}

\subsection*{Code}
After determining the control function $F = F_{\text{fr.comp.}}(\vartheta, \omega, v_x, b)$, locate the following function in the code:
\begin{verbatim}
def cartpole_energy_based_control_function_friction_compensation(
    self,
    angle: float, 
    angle_vel: float, 
    vel: float, 
    friction_coeff: float, 
) -> float:
\end{verbatim}
and implement the function body so that it computes and returns $F_{\text{fr.comp.}}(\vartheta, \omega, v_x, b)$. The variable correspondences in the code are as follows:
\begin{itemize}
    \item $\vartheta=$\texttt{angle}
    \item $\omega=$\texttt{angle\_vel}
    \item $v_x=$\texttt{vel}
    \item $b=$\texttt{friction\_coeff} 
    \item $k=$\texttt{self.energy\_gain}
    \item $\lambda=$\texttt{self.velocity\_gain}
    \item $m_c=$\texttt{mass\_cart}
    \item $m_p=$\texttt{mass\_pole}
    \item $l=$\texttt{length\_pole}
    \item $g=$\texttt{grav\_const}
\end{itemize}

\section*{Exercise 2}

\subsection*{Theory}
Determine the function $B(\vartheta, \omega, v_x)$ such that the time derivative $\frac{\diff L_2}{\diff t}$ of
$$
L_2 = \frac{1}{2}\left(E_p^2 + m_p l\lambda v_x ^ 2 + \frac{(\hat{b} - b) ^ 2}{\alpha}\right), \text{ where } \hat{b} = \hat{b}(t) = \hat{b}(0) + \alpha\int_{0}^{t} B(\vartheta, \omega, v_x) d \tau
$$
is equal to
$$
\frac{\diff L_2}{\diff t} = - m_p lk (E_p\omega \cos \vartheta  - \lambda v_x)^2.
$$
This should be valid under the control law given by
$$
    F = F_{\text{fr.comp.}}(\vartheta, \omega, v_x, \hat{b})
$$
In the above equations, $\hat{b}(0)$ is known and equals zero, and  $\alpha \in \R_{>0}$ is a positive constant hyperparameter.
\\
\\
\textbf{Hint.} \textit{Refer to the appendix for a similar deduction applied to the inverted pendulum system. The appendix should be reviewed with attention for guidance.}  

\subsection*{Code}
After deriving the function $B(\vartheta, \omega, v_x)$, locate the following function in the code:
\begin{verbatim}
def euler_update_friction_coeff_estimate(
    self,
    angle: float, 
    angle_vel: float, 
    vel: float,
) -> None:
\end{verbatim}
and complete its definition to update $\hat{b}$ using the Euler method:
\begin{equation*}
    \hat{b} := \hat{b} + \alpha B(\vartheta, \omega, v_x) \Delta t
\end{equation*}
The variable correspondences in the code are as follows:
\begin{itemize}
    \item $\alpha=$\texttt{self.friction\_coeff\_est\_learning\_rate}
    \item $\hat{b} =$\texttt{self.friction\_coeff\_est}
    \item $\Delta t =$\texttt{self.sampling\_time}
    \item $k=$\texttt{self.energy\_gain}
    \item $\lambda=$\texttt{self.velocity\_gain}
    \item $m_c=$\texttt{mass\_cart}
    \item $m_p=$\texttt{mass\_pole}
    \item $l=$\texttt{length\_pole}
    \item $g=$\texttt{grav\_const}
\end{itemize}

\newpage
\section*{Appendix}

\subsection*{Derivation of the Adaptive Controller for Inverted Pendulum System with Friction}

Consider the inverted pendulum system with friction described by the state dynamics:
\begin{equation}
    \label{eqn_state_dynamics_inv_pendulum}
    \begin{aligned}
        &\dot{\vartheta} =  \omega \\
        &\dot{\omega} = \frac{g}{l}\sin \vartheta + \frac{M}{ m l ^ 2} - b \omega ^ 2 \sgn (\omega)
    \end{aligned}
\end{equation}
where
\begin{itemize}
    \item $\vartheta$ is the pendulum angle (\textbf{state variable}) [rad]
    \item $\omega$ is the pendulum angular velocity (\textbf{state variable}) [rad/s]
    \item $M$ is the pendulum torque (\textbf{control variable}) [kg$\times\text{m} ^ 2$/$\text{s} ^ 2$]  
    \item $m$ is the pendlum mass [kg]
    \item $l$ is the pendulum length [m]
    \item $g$ is the gravity constant [m/$\text{s}^2$]
    \item $b$ is the friction coefficient [$\text{m} ^ {-2}$] 
\end{itemize}
We define the Lyapunov function $L$ as:
$$
    L = \frac{1}{2} \left(E^2 + \frac{(\hat{b} - b) ^ 2}{\alpha}\right) 
$$
where
\begin{itemize}
    \item $E = \frac{m l ^2 \omega ^ 2}{2} + mgl(\cos\vartheta - 1)$ is the energy of the system,
    \item $\alpha \in \R_{>0}$ is a positive constant hyperparameter, interpreted as a learning rate,
    \item $\hat{b} = \hat{b}(t)$ is an estimate of the real friction coefficient $b$. The function $\hat{b}$ is derived below, aiming \textbf{to find a control law and estimate for $\hat{b}$ that are independent from the actual value of $b$, while ensuring that $\dot{L} \leq 0$}. 
\end{itemize}
The derivative of $L$ is given by:
$$
    \dot{L} = E \dot{E} + \frac{\hat{b} - b}{\alpha} \dot{\hat{b}}
$$
Examining $\dot{E}$ more closely:
\begin{multline*}
    \dot{E} = ml^2 \omega \dot{\omega} - mgl \omega \sin\vartheta  = \omega (mgl \sin \vartheta + M - b ml^2 \omega ^ 2 \sgn(\omega)  - mgl \sin \vartheta) = 
    \\
    = M\omega - b ml^2 |\omega| ^ 3
\end{multline*}
Hence,
\begin{multline*}
    \dot{L} = E (M\omega - b ml^2 |\omega| ^ 3) +  \frac{\hat{b} - b}{\alpha} \dot{\hat{b}} = 
    \left\{\text{Let us add and subtract} \pm  E\hat{b}ml^2 |\omega| ^ 3\right\} = \\ 
    = E (\omega M - \hat{b} ml^2 |\omega| ^ 3) + \left(\hat{b} - b\right)\left(E ml^2 |\omega| ^ 3 + \frac{\dot{\hat{b}}}{\alpha} \right)
\end{multline*}
By setting:
\begin{itemize}
    \item $\dot{\hat{b}} = - \alpha E m l ^ 2 |\omega| ^ 3$, which implies  $\hat{b}(t) = \hat{b}(0) +  \alpha \int_0^t - E m l ^ 2 |\omega| ^ 3 \diff \tau $
    \item the control law $M = - k \sgn(\omega E) + |\omega|\omega \hat{b} m l ^ 2$  for some positive constant $k \in \R_{> 0}$
\end{itemize}
we ensure that:
$$
    \dot{L} = - k |w E| \leq 0
$$
\subsection*{Important Remark}
It is straightforward to verify that:
\begin{equation}
    \label{eqn_cotrol_law_inv_pendlum_friction_compenstaion}
    \frac{\diff }{\diff t} \left(\frac{E ^ 2}{2}\right) =  - k |w E| \text{ for }  M = - k \sgn(\omega E) + |\omega|\omega b m l ^ 2
\end{equation}
but to practically apply this control law, one must know the true value of the friction coefficient $b$. In contrast, we previously derived an adaptive control law of similar form,
\begin{equation}
    \label{eqn_cotrol_law_inv_pendlum_friction_adaptive}
    M = - k \sgn(\omega E) + |\omega|\omega \hat{b} m l ^ 2.
\end{equation}
The only difference between the control laws \eqref{eqn_cotrol_law_inv_pendlum_friction_compenstaion} and \eqref{eqn_cotrol_law_inv_pendlum_friction_adaptive} is the substitution of $b$ with its estimate $\hat{b}(t) =\hat{b}(0) +  \alpha \int_0^t - E m l ^ 2 |\omega| ^ 3 \diff \tau $. \textbf{This estimate can be computed in practice using the Euler scheme, without requiring knowledge of the actual value of the friction coefficient $b$}.

\end{document}